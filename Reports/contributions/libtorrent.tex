% tweede zin herschrijven
% weglaten van de russische libtorrent
\section{Libtorrent}
	Our work was greatly dependent on one of the worlds most popular libraries: libtorrent. Much effort has gone into the cross-compilation of libtorrent and we are relieved that we now have a stable version of the library compiled for the ARM architecture of Android devices. Since so much time and effort has gone into libtorrent, we decided to create a separate tutorial about the compilation process. This tutorial is available as an appendix. This means that other people can extend on our work and use our building process to compile libtorrent for their own purposes. In the future, a recipe of libtorrent can be made that performs the compilation process.
	
	% label erbij
	% weg of niet
	When compiling libtorrent, we used the Boost library that provides advanced C++ features such as memory management and support for multi threading. libtorrent-rasterbar was used to create our libtorrent library we can use in Python for Android. During the project, we also tried a Russian version of libtorrent that Jaap, a member of the Tribler group, successfully used in his bachelor project. This version was quite different from the official libtorrent. Jaap already showed his interest in our compiled version and he is going to use it in his application.