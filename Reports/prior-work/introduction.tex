\section{Introduction}
	The increasing use of the Internet had its downsides. Cybercrime is a recent and growing problem: we are not always safe when we are browsing the worldwide web. Scamming and hacking are serious problems of the 21th century. In some countries like China and North Korea, the Internet is censored and inhabitants are prohibited from visiting websites that are forbidden by the government. Anonymous communication could be a solution for these countries. A popular network for anonymous communication is Tor, however, as we will see Tor has many problems and does not scale very well. Another step in the right direction is Tribler, an anonymous peer-to-peer network created by the TU Delft. The rise of the Internet also brings discussion about net neutrality, which is a highly debated issue in many countries.
	
	Since the introduction of the Internet in 1972\cite{leiner1997past}, we cannot imagine a life without it. We make extensive use of applications like Facebook, Twitter and Netflix, applications that have only recently emerged. We are using the Internet to share moments of our life, to run our business or for our entertainment by playing online games or watching videos. The innovation continues at a high rate and we are using the Internet more and more.
	
	We are not only connected with a desktop computer or laptop anymore: smartphones are raising in popularity. Mobile devices like smartphones and tablets allows us to be always connected with each other. Popular mobile applications like WhatsApp and Instagram often have millions, if not billions of users \cite{googleplayinstagram, googleplaywhatsapp}.
	
	In this research, we will first describe the Python for Android framework, and why we are using it. Followed by section \ref{scc:motivation}, we motivate the move to the mobile platform Android. After that, we will describe the Tor network in section \ref{scc:tor} and look into its advantages and disadvantages. Next, the focus will move to Tribler and we will discuss the Tribler system in section \ref{scc:tribler}. In section \ref{scc:p4a}, we will discuss a library that allows to run Python code on an Android device: Python for Android. One of the contributers to Python for Android is The Global Square. We will explain what The Global Square exactly is and what their goals are in section \ref{scc:tgs}. Finally, we will discuss how we will be using the discussed software for our projects and we will elaborate what our project exactly is and what it does.