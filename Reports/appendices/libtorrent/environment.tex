\section{Setting up the environment}
	To cross compile libtorrent for Android, we first need to create a custom toolchain. Before we do that, please make sure you have the Android NDK and SDK installed. We compiled with NDK r9d 32-bit and we also installed SDK version 14. We are not sure whether other versions of the NDK / SDK are working. The compilation has been executed on a 64-bit Ubuntu 14.04 machine. Please note that if you are compiling on a 64-bit machine, you need some 32-bit support packages which can be installed with the following command.
	\begin{lstlisting}
sudo apt-get install ia32-libs
	\end{lstlisting}
	This should install the required 32-bit libraries needed for the cross-compilation. The next step is to install Python for Android because we need the Python library from it. We cannot use the Python distribution the system is using because that is compiled for an incompatible architecture. The Python for Android framework comes with a distribute script that creates a distribution with the supplied libraries. Executing the following commands from the Python for Android folder should build the Python distribution that will eventually run on the device.
	\begin{lstlisting}
export ANDROIDSDK='<path to your SDK folder>'
export ANDROIDNDK='<path to your NDK folder>'
export ANDROIDNDKVER=r9d
export ANDROIDAPI=14
./distribute.sh -m kivy
	\end{lstlisting}
	An additional step is required: copy the \emph{pyconfig.h} file to the \emph{Include} folder in the build directory of \emph{pythoninstall/Python2.7.2}. This file is required by the compilation process of the Python bindings.\\
	The next step is to install the custom toolchain we create from the Android NDK. Execute the following command. Your toolchain location can be anywhere on the computer but it is recommended to install it in a location where you can access it easily.
	\begin{lstlisting}
<path to your NDK folder>/build/tools/make-standalone-toolchain.sh \
--platform=android-14 --install-dir=<your new toolchain location>
	\end{lstlisting}
	This command should generate a custom toolchain in the location specified by the install-dir argument.\\
	Several export variables are required for compiling Boost and libtorrent, these are listed below. The \emph{\$ANDROIDNDK} variable should already be defined from the previous steps.
	\begin{lstlisting}
# Custom paths
export ANDROIDNDK='<path to your NDK folder>'
export SYSROOT=$ANDROIDNDK/platforms/android-14/arch-arm
export PYTHON= \ 
<path to your Python for Android>/build/python/Python-2.7.2/hostpython
export PYTHON_CPP_FLAGS= \
"-I<path to your Python for Android>/build/python/Python-2.7.2 \
 -I<path to your Python for Android>/build/python/Python-2.7.2/Include"
	 
# Custom ARM toolchain
export SYSROOT=$ANDROIDNDK/platforms/android-14/arch-arm
export PATH=/usr/local/gcc-4.8.0-arm-linux-androideabi/bin:$PATH
export CC=arm-linux-androideabi-gcc
export CXX=arm-linux-androideabi-g++
export CROSSHOST=arm-linux-androideabi
export CROSSHOME=/usr/local/gcc-4.8.0-arm-linux-androideabi
	\end{lstlisting}
	It is recommended to create a shell script file with these exports and load it using the source command.\\
	You should now have a custom toolchain ready to be used and the right environment variables setup.