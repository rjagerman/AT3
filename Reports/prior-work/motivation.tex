\section{Motivation}
	Worldwide the amount of mobile devices is growing fast. In 2013, there were 6.8 billion mobile subscriptions worldwide \cite{itustatistics}. This number is expected to exceed the world population in 2014 with 7.3 billion subscriptions. Android had a share of 81\% in Q3 of devices shipped \cite{forbesandroidmarket}.

	This shows that there is a lot of potential gain on the Android market. Android is open source and can be modified more than a platform like iOs, on top of that is the Google Play not subject to a review before you can release your application, which allows for an easier deployment. Since a library called Python for Android already exists, it's also a more natural choice to develop it for Android since as stated earlier, the Tor-tunnel functionality is written in Python.

	Besides the arguments stated above, a mobile application is an excellent way to attract more new users to Tribler. While the Tor project does have an Android app named Orbot, this app does not provide anonymous file transfer \cite{tororbot, googleplayorbot}. Orbot only serves as an anonymous proxy. On top of that, the structure of Tor is currently semi-centralized \cite{jagerman2014fifteen} where this app will be completely decentralized using Dispersy as synchronization system \cite{zeilemaker2013dispersy}. With the billions of mobile devices out there and no competition in this area, this app has a lot of potential to open a new way of anonymous file sharing while being on the go.
	
