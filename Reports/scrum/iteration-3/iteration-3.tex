\section{Iteration 3}
\label{iteration2}
	In this section we describe our third sprint, starting in week 4.5 (19$^{th}$ of May) and ending in week 4.6 (30$^{th}$ of May).

	\subsection{Goals}
		This sprint, we had the following goals:
	
		\begin{enumerate}
			\item TODO
		\end{enumerate}
		
	\subsection{RUTracker libtorrent}
		Since we would like to have a stable libtorrent library that downloads a torrent without segmentation faults, we decided to try out the libtorrent that Jaap used in his Bachelor thesis. This is most probably an older version of libtorrent but has proven to be succesful in his project. Jaap provided us with a link to a Github page that contains instructions on how to build libtorrent from source (https://github.com/pepibumur/Torrent-Movies). This libtorrent version is used in an Android application called RUTracker, an Android application that allows users to download and upload torrent files in the background. We thought it would be a good idea to try out this libtorrent because it seemed to be quite stable.
		
		Instead of using a custom toolchain like we did to compile libtorrent-rasterbar, we used the toolchain that ships with the NDK. The first step was to build some Boost libraries libtorrent depends on (Boost.filesystem, Boost.system and Boost.thread). We used the Boost for Android project (https://github.com/MysticTreeGames/Boost-for-Android) to build Boost 1.49. After that, we compiled libtorrent according to the instruction given by the Torrent Movies project. We linked against the Boost libraries we just compiled and we got a shared object file we can use in our Android application.
		
			\subsubsection{Python bindings}
				After writing a small example to test if our compiled libtorrent library is stable, we came to the conclusion that it does not crash. Since this libtorrent version looks promising, we delved into the Python bindings we need to communicate between Python and a native C library. Importing this library without Python bindings, results in an error that an initializer function could not been found.
		
				For the Python bindings, it is convenient to use the Boost.python library. This library contains several macro's and methods to easily define our Python calls. The macro BOOST\_PYTHON\_MODULE, declared in boost/python.hpp, initializes our Python library and makes it ready for an import in Python. However, when running a minimal Python script that only imports libtorrent, a segmentation fault is thrown. This means that we are unable to use this version of libtorrent in Python. We're not sure what's the cause of this error. If the initialization and import of the library would work correctly, we could write our own bindings for the libtorrent functions and methods we need in Tribler.
			
	\subsection{Conclusion}
		Hier komen onze conclusies over deze sprint (wat ging er goed/fout, wat willen we anders doen, wat gaan we volgende sprint doen etc.)
