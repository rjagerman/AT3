\section{Iteration 1}
	In this section we describe our first sprint, starting in week 4.3 (5$^{th}$ of may) and ending in week 4.5 (16$^{th}$ of may).

	\subsection{Goals}
		This sprint, we had the following goals:
	
		\begin{enumerate}
			\item Get python code working on Android. Evaluate and set up Python for Android.
			\item Get all packages that are needed to run the anonymous tunnels working on Android.
			\item Implement a basic GUI to test with. This GUI should be created with Kivy.
		\end{enumerate}
	
	\subsection{Python for Android}
		The Python for Android framework allows developers to add existing Python packages by creating so called recipes. As most of the packages were not available or were custom made, we had to build a lot of these recipes ourselves.
	
		A recipe downloads a package, extracts the contents and applies operations on them when required, written by the developer himself.
	
		Python for Android allows to build these packages under the ARM architecture so they won't compile for the desktop architecture which is not suitable for Android development. By simply using the 'push arm' and 'pop arm' commands in the recipe files this can be achieved. Because of this easy integration and the fact that Tribler and the code for the anonymous tunnels was written in Python, this is a good environment to develop our application.
	
		Below are the packages described we use in our application and in most cases had to create a recipe for. We describe the functionality of the package in our application and which dependencies this package has.
	
		\begin{itemize}
		
			\item Kivy\\
			We use Kivy as framework for creating the GUI. Kivy is an open source software library for creating Natural User Interface (NUI) applications. It's easy to use and cross-platform, allowing users to create a GUI on their PC and then integrate it in their products (we integrate it in our Android application). As Kivy is integrated in Python for Android, it's a natural choice to use it for create the GUI. Kivy is dependent on Python, as it's a Python package. 
		
			\item OpenSSL\\
			OpenSSL is the world's most famous open source cryptography toolkit, also available for Python. As our application makes use of PyCrypto and M2Crypto which are both dependent on openSSL, we have to include it in our app.
		
			\item M2crypto\\
			Dispersy and Tribler are dependent on M2Crypto as it has some security features which M2Crypto implements such as elliptic curves cryptography. As the anonymous tunnels, our core function of the application, are dependent on both Dispersy and Tribler we also need M2Crypto. M2Crypto itself is dependent on Python and OpenSSL as it's a Python package using the OpenSSL implementation.
		
			\item PyCrypto\\
			PyCrypto is a Python library which implements certain cryptography functions used by the tribler\_core\_minimal package. PyCrypto itself depends on functionality of the OpenSSL package.
		
			\item Boost\\
			The Libtorrent library is used to download torrent files with the Tor protocol. To compile Libtorrent, the boost package is required. Boost is a huge C++ library which adds code for multithreading, regex, math and asynchronous operations. We have to compile Libboost with Python bindings enabled so we can invoke the library from Python code.
		
			\item Netifaces\\
			Netifaces allows to easily get the address(es) of the machine's network interfaces from Python. It's therefore dependent on Python and used by the Dispersy package.
		
			\item Zope\\
			Zope is a open source web framework for object-oriented web application servers. The Twisted package makes use of this framework for asynchronous networking.
		
			\item Twisted\\
			Twisted is an extensible framework for asynchronous networking written in Python. The framework has special focus on event-based network programming and multi-protocol integration. It has dependencies on the Zope framework.
		
			\item Anontunnels\\
			This package is the core of our application and contains the code needed for the anonymous tunnels. This package actually contains another package: the support for the Socks5 proxies. The files for this package come from pull request 525 on the Tribler Github.
			
			We made some minor changes to this code: the Main.py file has been removed from the package and the class definition of AnonTunnel has been moved to it’s own file (atunnel.py). We import this file in our main.py so we can use the AnonTunnel class. We also adjusted the master key in community.py so we have our own community to test communication between devices.
			
			\item Tribler\_core\_minimal\\
			The anonymous tunnels are using some parts of the core of Tribler. These files have been bundled in the tribler\_core\_minimal package. We’ve looked closely to the various imports the code for the anonymous tunnels is using and when a script imports a script from the Tribler core, it is added to the tribler\_core\_minimal package. For example, this package contains the RawServer and the SocketHandler classes. It also contains the code needed for the cryptography such as support for elliptic curve cryptography and ELGamal.
			
			\item Dispersy\\
			Since the code of the anonymous tunnels are using Dispersy for node discovery and data synchronization, we’ve created a Python package with all the code that’s needed for Dispersy. This package does not depend on other custom packages we made and can be used standalone.\\
			The files we have bundled are from pull request 525 on the Tribler Github. We didn’t use the files from the official Dispersy Github because this build was missing some classes we needed (for example, the decorator.py). Besides that, some changes have been made to the Dispersy core to add support for the anonymous tunnels.
		
		\end{itemize}
	
	\subsection{Porting the anonymous tunnels to Android}
		\textbf{TO DO: }In deze subsectie iets over de anontunnels die we hebben geport naar android, wat over Python for Android en de packages. Misschien de packages in subsubsections bespreken?
		
	\subsection{Attempt (?) to compile Libtorrent for Android}
		\textbf{TO DO:} Hier iets over onze uitdaging om Libtorrent werkend te krijgen in Python for Android
	
	\subsection{Creating a GUI with Kivy}
		The first version of the anonymous tunnels used the standard output for printing information about what’s going on. This required the phone to be connected to a computer so we can examine the log with the adb logcat tool. That’s why we decided to create a graphical user interface for our application. The purpose of this application is to provide a button to start the tunneling and a log to see on the screen what’s happening.
	
		Creating a GUI was a small stap for us: we already included the Kivy package in our Python for Android distribution. Since Kivy is a GUI framework for desktop (but has been ported to Python for Android), we first tried to create a desktop interface. Creating interfaces in Kivy is quite straightforward and is related to creating user interfaces in Android: you specify your layout elements in Kivy files which have the kv extension. In the main Python file, you load this interface file and you can access properties of the UI elements.
	
	\subsection{Conclusion}
		Hier komen onze conclusies over deze sprint (wat ging er goed/fout, wat willen we anders doen, wat gaan we volgende sprint doen etc.)
